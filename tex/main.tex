\documentclass[10pt,a4paper]{article}
%\usepackage[utf8]{inputenc}
%\usepackage[T1]{fontenc}
%\usepackage[english]{babel}

\usepackage{cermics}
\usepackage{graphicx}
\usepackage{subfig}

\begin{document}
	
\title{\textbf{Toy Energy Management System}}
\author{Adrien Le Franc}
\date{}
\maketitle

\section{Introduction}

We consider a simple Energy Management System (EMS) for an electrical network composed with a solar panel, a battery, a connexion to the global network and an electrical demand. The aim of this system is to operate the battery so as to minimize the energy importation over a time period \horizon, while dealing with the uncertain energy demand and solar production. This goal formulates as a stochastic optimization problem. We first introduce notations for the problem formulation. All capital letters will denote random variables (e.g. $X$) while lower case will be devoted to realizations of random variables (e.g. $x$).

\begin{itemize}
\item $\State_t$ is the battery stock at time $t$
\item $\Control_t$ is the energy exchanged from the system to the battery. If the realization $\control_t$ is positive (negative) the battery is being charged (discharged) over the time segment $\nc{t-1 \eqsepv t}$
\item $\Uncertain_{1, t}$ is the energy produced by the solar panel over the time segment $\nc{t-1 \eqsepv t}$
\item $\Uncertain_{2, t}$ is the energy demand over the time segment $\nc{t-1 \eqsepv t}$
\end{itemize}

\noindent The energy is measured in kilowatt-heure, the time horizon for this problem is 24h and decisions are taken every $\Delta t$ = 15 min. The battery is controlled over \horizon = 96 consecutive time intervals in the \textit{hazard-decision} framework, which means that controls $\Control_t$ are decided after observing the corresponding noise $\Uncertain_t$ for $t \in \{1...T\}$. We consider a 13,5 kWh battery which restricts the state space to $\XX_t = \nc{0 \eqsepv 13,5}$. The dynamics of the system is as follows:
\[\State_{t+1} =  \dynamics(\State_t \eqsepv \Control_{t+1}) = \State_t + \rho_c \Control_{t+1}^+ - \rho_d^{-1} \Control_{t+1}^-\]
where $\rho_c$ ($\rho_d$) is a charge (discharge) efficiency coefficient. This dynamics restricts control spaces $\UU_t$ to admissible values which respect $\state_t \in \nc{0 \eqsepv 13,5}$. Over any time episode $\nc{t \eqsepv t+1}$ the EMS imports $E_{t+1}$ kWh to compensate other energy flows:
\[E_{t+1} + \Uncertain_{1, t+1} - \Uncertain_{2, t+1} - \Control_t = 0\]
Thus, using a single noise variable $\Uncertain_t = \Uncertain_{2, t} - \Uncertain_{1, t}$ and considering the (deterministic) energy price $(p_t)_{t=0..\horizon-1}$, the cost to pay for episode $\nc{t \eqsepv t+1}$ is:
\[L_t(\Control_{t+1} \eqsepv \Uncertain_{t+1}) = p_t \times \Max \big(0 \eqsepv \Uncertain_{t+1} + \Control_{t+1} \big)\]

\noindent The non-anticipativity constraint imposes that the decision maker does not access information about the realizations of future noises. Given the filtration $\mathcal{F}_t = \sigma(\Uncertain_1, ..., \Uncertain_t)$, this constraint is expressed by $\sigma(\Control_t) \subset \mathcal{F}_t$. From the Doob lemma (\cite{carpentier2015stochastic} chapter 3.4.4), it is equivalently formulated 
\[\Control_t = \pi_t(\Uncertain_1, ..., \Uncertain_t) \]
\noindent where $(\pi_t)_{t=1..\horizon}$ are policy functions which we restrict to $\pi_t(\State_{t-1}, \Uncertain_t)$ in this work.

\vspace{0.5cm}

\noindent Eventually, the stochastic optimization problem that we solve formulates as follows:

\begin{equation}
\begin{aligned}
& \Min_{\pi_1 ... \pi_\horizon} 
& & \besp{\sum_{t=0}^{\horizon-1} L_t(\Control_{t+1} \eqsepv \Uncertain_{t+1})} \\
&&& \State_{t+1} =  \dynamics(\State_t \eqsepv \Control_{t+1}) \\
&&& 0 \leq \State_t \leq 13,5 \\
&&& \Control_t = \pi_t(\State_{t-1}, \Uncertain_t)
\end{aligned}
\label{eq1.1}
\end{equation}

\section{Stochastic Dynamic Programming}

Stochastic dynamic programming (SDP) introduces $\textit{value functions}$ defined as the cost-to-go given a time and state position:
\[ V_t(x) = \Min_{\pi_{t+1} ... \pi_\horizon} \besp{\sum_{s=t}^{\horizon-1} L_s(\Control_{s+1} \eqsepv \Uncertain_{s+1}) | \State_t = x} \]

\noindent Assuming that noises $\np{\Uncertain_1, ..., \Uncertain_T}$ are independent, such functions are dynamically connected in time by the $\textit{Bellman equation}$
\begin{equation}
\begin{aligned}
V_T(x) &=  0  \\
V_t(x) &= \Min_{\pi_{t+1}} \ \mathbb{E}_{\mu_{t+1}^{off}} \bc{L_t(\Control_{t+1} \eqsepv \Uncertain_{t+1}) + V_{t+1}(x + \rho_c \Control_{t+1}^+ -\rho_d^{-1} \Control_{t+1}^-)} \\
&= \mathbb{E}_{\mu_{t+1}^{off}} \bc{ \Min_u \ba{L_t(u \eqsepv \Uncertain_{t+1}) + V_{t+1}(x + \rho_c u^+ -\rho_d^{-1} u^-)}} 
\end{aligned}
\label{eq2.1}
\end{equation}

\noindent Here $\mu_{t+1}^{off}$ designs an offline (training) probability measure on the $\sigma$-algebra $\mathcal{W}_{t+1}$ attached to the noise space $\UNCERTAIN_{t+1}$. Equation \eqref{eq2.1} is used to compute recursively the value functions $\bp{V_t(x)}_{t \in {1..T}, x \in \STATE}$. After this offline backward computation step, optimal policies are computed online given an initial state $x_0$ and observing noisy realizations $w_t$ before deciding on the control $u_t$:
\[ \pi_t(x_{t-1} \eqsepv w_t) \in \argmin_u{\ba{L_{t-1}(u \eqsepv w_t) + V_t(x + \rho_c u^+ - \rho^{-1}_d u^-)}} \]

\section{Risk Averse SDP}

\subsection{Average value at risk}

\noindent The $\alpha$ value at risk of a real value random variable $X$ is the tightest upper bound on the realizations of $X$ which is satisfied with probability at least $1-\alpha$:
\[ V@R_{\alpha}\nc{X} = \inf \na{t: \ \nprob{X \leq t} \geq 1-\alpha } \] 

\noindent Constraining $V@R_\alpha\nc{X}$ enables to penalize worst-case scenarios and is sometimes preferred to plain expectation minimization. However imposing $V@R_\alpha \leq \tau$ is equivalent to enforcing $\mathbb{E}\nc{\boldsymbol{1}_{\na{\tau, \infty}}(X)} \leq \alpha$ which reveals that the constraint is non-convex. The $\alpha$ average value at risk \footnote{also called conditional value at risk $CV@R_\alpha[X]$} is defined as 
\[ AV@R_\alpha\nc{X} = \inf \na{t + \alpha^{-1} \mathbb{E}\nc{X-t}_+ } \]
\noindent Imposing $AV@R_\alpha\nc{X} \leq \tau$ is a conservative convex approximation of the previous constraint (\cite{shapiro2009lectures} chapter 6.2.4). $AV@R_\alpha[X]$ can be expressed explicitly from the value at risk:
\[ AV@R_{\alpha}[X] = V@R_\alpha[X] + \alpha^{-1}\mathbb{E}\nc{X-V@R_\alpha[X]}_+ = \mathbb{E}\nc{X|X \geq V@R_\alpha[X]} \]

\noindent We then build a new risk measure defined as a weighted sum of the expectation and the average value at risk:
\[ \rho_\lambda\nc{X} = (1-\lambda) \mathbb{E}\nc{X} + \lambda AV@R_\alpha\nc{X} \]

\noindent A major advantage of this risk measure is that it is a coherent risk measure (\cite{shapiro2009lectures} chapter 6.3) and that we can write dynamic programming equations to solve the nested risk averse stochastic optimization problem arising when switching expectations with $\rho_\lambda$ in \eqref{eq1.1}

\subsection{Dynamic programming}

\noindent In order to write a risk-averse formulation of \eqref{eq1.1} we take a backward approach of the multi stage problem. When observing $\uncertain_T$ in state $x_{T-1}$, taking the last decision $\control_T$ formulates as follows:
\[ \Min_{\control \in \mathcal{U}(x_{T-1})} L_{T-1} \np{\control, w_\horizon} \]

\noindent In the previous stage, the goal is to minimmize the immediate cost $L_{T-2}$ while minimizing the cost to go evaluated with the risk measure $\rho_\lambda$ and a probability measure $\mu_T$ on the $\sigma$-algebra $\mathcal{W}_T$ attached to the noise space $\UNCERTAIN_T$:
\[ \Min_{\control \in \mathcal{U}(x_{T-2})} L_{T-2} \np{\control, w_{\horizon-1}} + \rho_\lambda^{\mu_T} \bc{\inf_{\control_T \in \mathcal{U}(x_{T-1})} L_{T-1} \np{\control_T, w_\horizon}}\]

\noindent Eventually,  the global risk-averse EMS problem can be written in a nested formulation:
\begin{equation} 
\Min_{\control \in \mathcal{U}(x_0)} L_0 \np{\control, w_1} + \rho_\lambda^{\mu_2} \Bc{ \inf_{\control_2 \in \mathcal{U}(x_1)} L_1 \np{\control_2, w_2} + ... + \rho_\lambda^{\mu_T} \bc{\inf_{\control \in \mathcal{U}(x_{T-1})} L_{T-1} \np{\control, w_\horizon}} }
\label{eq3.1}
\end{equation}

\noindent We can use dynamic programming to compute the value functions of the risk averse problem:
\begin{equation}
\begin{aligned}
V_T(x) &=  0  \\
\tilde{V}_t(x) &= \Min_{u \in \mathcal{U}(x)} \ L_t(\control \eqsepv \uncertain_{t+1}) + V_{t+1}(x + \rho_c \control^+ -\rho_d^{-1} \control^-) \\
V_t(x) &= \rho_\lambda^{\mu_{t+1}} \bc{\tilde{V}_t(x)} 
\end{aligned}
\label{eq3.2}
\end{equation}

\noindent and then compute online optimal policies in the same way as for the risk-neutral problem
\[ \pi_t(x_{t-1} \eqsepv w_t) \in \argmin_u{\ba{L_{t-1}(u \eqsepv w_t) + V_t(x + \rho_c u^+ - \rho^{-1}_d u^-)}} \]

\section{Comparative study}

\begin{figure}[h]
	\centering
	\subfloat[Risk neutral $\mu = 0.0157$]{{\includegraphics[width=5cm]{images/RN.png} }}
	\quad
	\centering
	\subfloat[Risk averse $\mu = 0.0162$]{{\includegraphics[width=5cm]{images/RA.png} }}
	\caption{histogram of daily cost for 1000 scenarios. Risk averse SDP with loss $\rho_\lambda\nc{X} = (1-\lambda) \mathbb{E}\nc{X} + \lambda AV@R_\alpha\nc{X}$, $\lambda = 0.2$, $\alpha = 0.25$}	
	\label{f222}
\end{figure}

\bibliographystyle{plain}
\bibliography{mybib}

\end{document} 